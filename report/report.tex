% !TeX spellcheck = en_US
% !TeX encoding = UTF-8
\documentclass[10pt, a4paper]{article}
\usepackage{graphics, graphicx, graphics}
\usepackage{fancyvrb, enumerate}
\usepackage{amsmath, amssymb, amscd, amsfonts}
\usepackage{geometry}
\usepackage{multirow}
\usepackage{url}
\usepackage{listings, listing}
\usepackage{color}
\usepackage{apacite}

\geometry
{
    top = 20mm,
    bottom = 20mm,
    left = 20mm,
    right = 20mm
}

\title{FID Data Reconstruction and $T_1$, $T_2$, $T_2^{*}$ Fitting}
\author{Jaewoong Lee}
\date{\today}

\begin{document}
    \maketitle
	\newpage
	
	\tableofcontents
	\listoftables
	\listoffigures
	\newpage
	
	\section{Theory}
		\subsection{Magnetic Resonance Imaging}
			Magnetic Resonance Imaging (MRI) is a medical imaging technique used in radiology to form pictures of the anatomy and the physiological process of the body. The basis of MRI techniques is the measurement of radio-frequency radiation resulting from transition induced between nuclear spin states of tissue hydrogen protons in the presence of a strong external magnetic field. \cite{ref:MRI1}
			
			The two parameters that are often used to characterize the behavior of an MRI signal and which can be used as a basis for generating MRI contrast are the spin-lattice (T1) and spin-spin (T2) relaxation times. \cite{ref:MRI1}
	
	\section{Method}
	
	\section{Results}
	
	\section{Discussion}
	
	\addcontentsline{toc}{section}{References}
	\bibliographystyle{apacite}
	\bibliography{reference}
\end{document}